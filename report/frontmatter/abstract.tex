\chapter*{Abstract}
\addcontentsline{toc}{chapter}{Abstract}
\makeatletter\@mkboth{}{Abstract}\makeatother

Automatic speech recognition (ASR) is an important technology for education \cite{wald2005using}, 
helping people with disabilities \cite{terbeh2013automatic}, and captioning (especially for deaf and hard of hearing people \cite{wald2006captioning}).
Several high quality ASR systems for English exist today, such as Siri, Alexa, and Google Assistant. However, ASR remains a challenge for low-resource languages. 

In this study, we perform ASR for two South African languages: Afrikaans and isiXhosa.
Our approach involves fine-tuning well-known pre-trained models, and using seperately trained language models to boost performance.
We propose a sequential fine-tuning approach, in which the model is first fine-tuned on one language, and then fine-tuned on the target language.
Our sequential fine-tuning approach is compared to a simple fine-tuning approach, where one language is used to fine-tune the model.
Our models are evaluated using the word error rate (WER).

Our experimental results conclude that...

% \selectlanguage{afrikaans}
% \chapter*{Uittreksel}
% \addcontentsline{toc}{chapter}{Opsomming}
% \makeatletter\@mkboth{}{Opsomming}\makeatother

% \hl{TODO}: Die Afrikaanse uittreksel.

% \selectlanguage{english}
