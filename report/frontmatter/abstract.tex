\chapter*{Abstract}
\addcontentsline{toc}{chapter}{Abstract}
\makeatletter\@mkboth{}{Abstract}\makeatother
Automatic speech recognition (ASR) involves identifying the spoken words for a given speech recording and returning the text of the spoken words.
ASR is an important technology used across many domains such as enhancing educational systems \cite{wald2005using}, 
helping people with disabilities \cite{terbeh2013automatic}, and automatic captioning \cite{wald2006captioning}.
Several high-quality ASR systems for English are developed such as Siri, Alexa, and Google Assistant.
However, creating high-quality ASR systems for low-resource languages remains a challenge.
In this study we perform ASR for two low-resource languages: Afrikaans and isiXhosa.

Our approach involves fine-tuning XLS-R for ASR.
XLS-R is a large-scale wav2vec 2.0 model that transforms speech recordings (audio data) into cross-lingual speech features.
Two fine-tuning strategies are compared in this study, referred to as the \emph{basic} fine-tuning strategy and the \emph{sequential} fine-tuning strategy.
Basic fine-tuning involves fine-tuning on one language, and sequential fine-tuning involves fine-tuning on two languages sequentially.
For our sequential fine-tuning experiments, we fine-tune on Dutch before fine-tuning on Afrikaans, and we fine-tune on isiZulu before fine-tuning on isiXhosa.
We create several ASR models using both fine-tuning strategies and our models are evaluated using the word error rate (WER) metric.

Our results conclude that the sequential fine-tuning strategy is more effective than the basic fine-tuning strategy for both Afrikaans and isiXhosa. 
By using sequential fine-tuning, our best Afrikaans model achieves a $0.3716$ WER on our Afrikaans test set, and our best isiXhosa model achieves a $0.4989$ WER on our isiXhosa test set.
By using a seperately trained $n$-gram language model (LM) we achieve a $0.2796$ WER for our best Afrikaans model, and a $0.3993$ WER for our best isiXhosa model.

% \selectlanguage{afrikaans}
% \chapter*{Uittreksel}
% \addcontentsline{toc}{chapter}{Opsomming}
% \makeatletter\@mkboth{}{Opsomming}\makeatother

% \hl{TODO}: Die Afrikaanse uittreksel.

% \selectlanguage{english}
