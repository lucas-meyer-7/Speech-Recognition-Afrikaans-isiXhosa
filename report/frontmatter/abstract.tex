\chapter*{Abstract}
\addcontentsline{toc}{chapter}{Abstract}
\makeatletter\@mkboth{}{Abstract}\makeatother
% 1. Motivation
Developing high-quality automatic speech recognition (ASR) systems for languages such as Afrikaans and isiXhosa remains a challenging task, mainly due to a limited number of available speech resources. 
% (TODO) Impact if succesful
% 2. Problem statement
In this study, we investigate the use of additional data from related languages in the development of ASR systems for Afrikaans and isiXhosa.
Specifically, we determine whether the use of Dutch data improves the performance of our Afrikaans systems
and whether the use of isiZulu data improves the performance of our isiXhosa systems.
We believe that while the use of Dutch and isiZulu data may improve performance, it also requires more computational resources to train our systems.

% 3. Approach
We compare two training strategies. The first strategy involves training on only the target language, and the second strategy 
involves training on the target language and a related language.
We train several Afrikaans and isiXhosa models using both strategies and evaluate the performance of each model on our own validation and test data.
Additionally, we use separately trained language models to improve the performance of 
our best models.
% 4. Results
We find that using additional data from a related language can slightly improve performance.
By only training on the target language, our best Afrikaans model achieves a $28.30\%$ word error rate (WER) on our Afrikaans test data 
and our best isiXhosa model achieves a $41.47\%$ WER on our isiXhosa test data.
By using additional Dutch and isiZulu data, our best Afrikaans model achieves a $27.50\%$ WER on our Afrikaans test data
and our best isiXhosa model achieves a $40.51\%$ WER on our isiXhosa test data.
% 5. Conclusion
We believe that using additional data from related languages could potentially provide a strategy for developing ASR systems in other under-resourced settings.
Code, examples, and models: \href{https://github.com/lucas-meyer-7/Speech-Recognition-Afrikaans-isiXhosa}{https://github.com/lucas-meyer-7/Speech-Recognition-Afrikaans-isiXhosa}.

% \selectlanguage{afrikaans}
% \chapter*{Uittreksel}
% \addcontentsline{toc}{chapter}{Opsomming}f
% \makeatletter\@mkboth{}{Opsomming}\makeatother

% \hl{TODO}: Die Afrikaanse uittreksel.

% \selectlanguage{english}
