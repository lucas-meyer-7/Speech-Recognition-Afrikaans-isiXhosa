\documentclass[conference]{IEEEtran}
\usepackage{tikz}
\usepackage{cite}
\usepackage{amsmath,amssymb,amsfonts,amsthm}
\usepackage{algorithm}
\usepackage{algpseudocode}
\usepackage{graphicx}
\usepackage{textcomp}
\usepackage{mathtools}
\usepackage{subcaption}
\usepackage{multirow}
\usepackage{physics}
\usepackage{xcolor}
\usepackage{siunitx}
\usepackage{float}
\usepackage[utf8]{inputenc}
\usepackage[english]{babel}
\usepackage{hyperref}
% \usepackage[hidelinks]{hyperref}



% ++++++++++++++++++++++++++++++++++++
% TITLE & AUTHOR
% ++++++++++++++++++++++++++++++++++++
\begin{document}
\title{Automatic cross-lingual speech recognition for Afrikaans and isiXhosa}
\author{
    \IEEEauthorblockN{\textbf{Lucas Meyer}}
    \IEEEauthorblockA{
        Supervisor: Herman Kamper\\
        \emph{Department of Applied Mathematics}\\
        \emph{University of Stellenbosch}\\
        Western Cape, South Africa\\
        \href{mailto:22614524@sun.ac.za}{22614524@sun.ac.za}\\
    }
}
\maketitle



% ++++++++++++++++++++++++++++++++++++
% ABSTRACT
% ++++++++++++++++++++++++++++++++++++
\begin{abstract}
    Abstract goes here ...
\end{abstract}



% ++++++++++++++++++++++++++++++++++++
% Introduction
% ++++++++++++++++++++++++++++++++++++
\section{Introduction}\label{sec:introduction}



% ++++++++++++++++++++++++++++++++++++
% Background
% ++++++++++++++++++++++++++++++++++++
\section{Background: Automatic Speech Recognition}\label{sec:background}
Automatic speech recognition (ASR) is a task that is intuitive and easy to understand: given a speech
recording of arbitrary length, predict the spoken words of the person in the recording. The difficulty
of ASR tasks depends on the data used to train ASR models. As is the case with many other machine learning models,
the quality of the data effects the robustness of the model.

The quality of speech data depends on the following factors:
\begin{itemize}
    \item The amount of available training data.
    \item The vocabulary size of the transcription text.
    \item Who the speaker in the recording is speaking to. Does the data contain \emph{conversational} or \emph{read} speech.
    \item Microphone position and Microphone quality.
    \item The presence of background noise.
    \item The accent of the speaker, which depends on the gender, age, and ethnicity of the speaker.
\end{itemize}

\section{Wav2Vec}



% ++++++++++++++++++++++++++++++++++++
% Methodology
% ++++++++++++++++++++++++++++++++++++
\section{Methodology}\label{sec:methodology}

\subsection{Data sets}
\subsubsection{Sadilar TTS data}
\subsubsection{FLEURS dataset}
\subsubsection{NCHLT dataset}

\subsection{Evaluation metrics}
\subsubsection{Word error rate}
The word error rate (WER) is equal to the number of character-level errors in the predicted transcript, 
divided by the number of words in the true transcript. One character-level error is corrected using one of three operations:
inserting a new character, deleting an existing character, or substituting an existing character for a new character.

\section{Emperical Procedure}


% ++++++++++++++++++++++++++++++++++++
% RESULTS
% ++++++++++++++++++++++++++++++++++++
\section{Results}\label{sec:results}



% ++++++++++++++++++++++++++++++++++++
% CONCLUSION
% ++++++++++++++++++++++++++++++++++++
\section{Conclusion}\label{sec:conclusion}



% ++++++++++++++++++++++++++++++++++++
% BIBLIOGRAPHY and EOD
% ++++++++++++++++++++++++++++++++++++
\begin{thebibliography}{1}

\end{thebibliography}
\end{document}