\documentclass[conference]{IEEEtran}
\usepackage{tikz}
\usepackage{cite}
\usepackage{amsmath,amssymb,amsfonts,amsthm}
\usepackage{algorithm}
\usepackage{algpseudocode}
\usepackage{graphicx}
\usepackage{textcomp}
\usepackage{mathtools}
\usepackage{subcaption}
\usepackage{multirow}
\usepackage{physics}
\usepackage{xcolor}
\usepackage{siunitx}
\usepackage{float}
\usepackage[utf8]{inputenc}
\usepackage[english]{babel}
\usepackage{hyperref}
% \usepackage[hidelinks]{hyperref}



% ++++++++++++++++++++++++++++++++++++
% TITLE & AUTHOR
% ++++++++++++++++++++++++++++++++++++
\begin{document}
\title{Automatic cross-lingual speech recognition for Afrikaans and isiXhosa}
\author{
    \IEEEauthorblockN{\textbf{Lucas Meyer}}
    \IEEEauthorblockA{
        Supervisor: Herman Kamper\\
        \emph{Department of Applied Mathematics}\\
        \emph{University of Stellenbosch}\\
        Western Cape, South Africa\\
        \href{mailto:22614524@sun.ac.za}{22614524@sun.ac.za}\\
    }
}
\maketitle



% ++++++++++++++++++++++++++++++++++++
% ABSTRACT
% ++++++++++++++++++++++++++++++++++++
\begin{abstract}
    Abstract goes here ...
\end{abstract}



% ++++++++++++++++++++++++++++++++++++
% Introduction
% ++++++++++++++++++++++++++++++++++++
\section{Introduction}\label{sec:introduction}



% ++++++++++++++++++++++++++++++++++++
% Background
% ++++++++++++++++++++++++++++++++++++
\section{Background}\label{sec:background}



% ++++++++++++++++++++++++++++++++++++
% Methodology
% ++++++++++++++++++++++++++++++++++++
\section{Methodology}\label{sec:methodology}

\subsection{Data sets}
\begin{itemize}
    \item Sadilar TTS data: Afrikaans and isiXhosa
\end{itemize}

\subsection{Preprocessing}

\subsection{Evaluation metrics}
\begin{itemize}
    \item Word error rate (WER): The word error rate is equal to the number of character-level errors in the predicted transcript, 
    divided by the number of words in the true transcript. One character-level error is corrected using one of three operations:
    inserting a new character, deleting an existing character, or substituting an existing character for a new character.
\end{itemize}



% ++++++++++++++++++++++++++++++++++++
% RESULTS
% ++++++++++++++++++++++++++++++++++++
\section{Results}\label{sec:results}



% ++++++++++++++++++++++++++++++++++++
% CONCLUSION
% ++++++++++++++++++++++++++++++++++++
\section{Conclusion}\label{sec:conclusion}



% ++++++++++++++++++++++++++++++++++++
% BIBLIOGRAPHY and EOD
% ++++++++++++++++++++++++++++++++++++
\begin{thebibliography}{1}

\end{thebibliography}
\end{document}