\graphicspath{{experimental_setup/fig/}}

\chapter{Experimental Setup} \label{chap:experimental_setup}

\section{Experiments}
\hl{TODO}:
\begin{itemize}
    \item Explain each experiment.
\end{itemize}

\section{Data sets}
\hl{TODO}:
\begin{itemize}
    \item Explain that we combined three different datasets to create two datasets for Afrikaans and isiXhosa.
    \item Provide huggingface links to datasets.
    \item Describe each of the three datasets with some very basic statistics such as counts and means.
    \item Explain how we made sure that our final datasets have similar duration histograms and total duration (just less than $7.5$ hours).
    \item Discuss preprocessing steps of audio ($\text{SR} = 16000$). Dicuss transcription preprocessing in LM subsection.
\end{itemize}

\subsection{Language model data}
\hl{TODO}:
\begin{itemize}
    \item Explain where we got the Wikipedia data.
    \item Cite WikiExtractor.
    \item Explain text preprocessing steps.
\end{itemize}


\section{Models and Hyperparameters}
\hl{TODO}: I am not really sure what to say here.

\section{Evaluation metrics}

\paragraph*{Word error rate}
The word error rate (WER) is equal to the number of character-level errors in the predicted transcript, 
divided by the number of words in the true transcript. One character-level error is corrected using one of three operations:
inserting a new character, deleting an existing character, or substituting an existing character for a new character.